\documentclass[a4paper,10pt]{article}
\usepackage[utf8]{inputenc}
\usepackage{listings}
\usepackage{color}
\usepackage{url}

\lstset{
language=Python,
basicstyle=\footnotesize,
numbers=left,
numberstyle=\tiny,
numbersep=5pt,
tabsize=1,
breaklines=true
}

%opening
\title{INFO-F-413 : Implémentation de l'Algorithme de Karger}
\author{Thomas Chapeaux}

\begin{document}

\maketitle

\section{Présentation}

\paragraph{}
L'algorithme de 
Karger\footnote{Réf. [1]}
est un algorithme probabiliste permettant d'estimer une coupe minimale d'un graphe,
basé sur la contraction d'arêtes choisies aléatoirement.

\paragraph{}
Au cours, il a été vu que cet algorithme a une probabilité de succès (c'est-à-dire de trouver effectivement une coupe minimale) strictement supérieure à
\begin{math} \frac{2}{n(n-1)} \end{math}.
Cette borne inférieure peut être améliorée jusqu'à
\begin{math} 1- \frac{1}{n^{2}} \end{math} en exécutant
\begin{math} n(n-1)\log{n} \end{math} itérations de l'algorithme et en gardant le meilleur résultat (la plus petite valeur trouvée).

\paragraph{}
Ce document présente une implémentation de l'algorithme en Python et tente de vérifier les valeurs trouvées au cours.

\section{Implémentation}

\subsection{Languages et bibliothèque}
\paragraph{}
L'algorithme a été implémenté en Python 2.7 à l'aide de la bibliothèque
igraph\footnote{http://igraph.sourceforge.net}.
Cette bibliothèque permet de générer des graphes aléatoirement via une méthode
géométrique\footnote{Réf. [2]}
et propose une méthode pour trouver la coupé minimale d'un graphe, ce qui permettra de juger de la qualité des valeurs trouvées par notre algorithme.
\paragraph{}
Python a surtout été choisi par préférence personnelle, mais également pour son efficacité lors de projets de taille raisonnable comme celui-ci.

\subsection{Implémentation de l'algorithme}
\subsubsection{Une itération}
\paragraph{} Lors de la contraction d'une arête A, chaque autre arête connectée à la destination de A est redirigée vers la source de
A\footnote{Le graphe étant non dirigé, on parle de source et de destination du point de vue de l'implémentation dans igraph},
et toutes les arêtes ayant la même source et la même destination que A (donc A également) sont supprimées.
\begin{lstlisting}
while (graph.vcount() > 2):
	edge_list = graph.get_edgelist()
	rand_edge_id = random.randint(0,g.ecount()-1)
	chosen_edge = edge_list[rand_edge_id]
	chosen_edge_source = chosen_edge[0]
	chosen_edge_target = chosen_edge[1]
	for e in edge_list:
		if (e[0] == chosen_edge_target or e[1] == chosen_edge_target):
			if (e[0] == chosen_edge_target and e[1] != chosen_edge_source):
				graph.add_edge(chosen_edge_source, e[1])
			elif (e[1] == chosen_edge_target and e[0] != chosen_edge_source):
				graph.add_edge(e[0], chosen_edge_source)
			edge_id = graph.get_eid(e[0], e[1])
			graph.delete_edges(edge_id)

	assert(graph.degree(chosen_edge_target) == 0)
	graph.delete_vertices(chosen_edge_target)
\end{lstlisting}
\subsubsection{Algorithme complet}
\paragraph{}


\end{itemize}




\section {Discussion}

\section{Références}

\begin{itemize}
  \item Karger, David (1993). "Global Min-cuts in RNC and Other Ramifications of a Simple Mincut Algorithm". Proc. 4th Annual ACM-SIAM Symposium on Discrete Algorithms. [1]
  \item ``n points are chosen randomly and uniformly inside the unit square and pairs of points closer to each other than a predefined distance d are connected by an edge'' \url{http://hal.elte.hu/~nepusz/development/igraph/tutorial/tutorial.html} [2]
\end{itemize}


\section{Annexes}

\subsection{Code source complet}
\lstinputlisting[language=Python]{../main.py}
\subsection{Exemple d'exécution}
ttt

\end{document}
